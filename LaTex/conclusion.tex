\chapter{Summary and Conclusion}

As stated already in the introduction, the main goal of this project is to understand more the nature of Jakarta floods by utilizing the data. There are in total four different chapters in this project:\\ 

\noindent
\textbf{In the first chapter}, the association between rainfall rate and flood occurrences in Jakarta's sub-districts is investigated. With the possible correlation between two variables, the suggestion regarding the best period of time in a year for the authorities to do precautionary and mitigation measurements is given.\\

\noindent
\textbf{In the second chapter}, two predictive modeling algorithm with regression model are built in order to predict or estimate the number of sub-districts and people that will be affected by floods with any given rainfall rate. The number of sub-districts that will be affected by floods can be estimated with linear regression model, while the amount of people who will be affected by floods can be estimated with third order polynomial regression model.\\

\noindent
\textbf{In the third chapter}, the districts in Jakarta is clustered into three segments in order to classify them based on how high their potential risks and severity should a heavy rainfall pours down Jakarta. With the clustering, it clears all of the doubts about which districts that the authorities should focus their attention to during heavy rainfall.\\ 

\noindent
\textbf{In the fourth chapter}, one possible solution to mitigate the floods, which is the amount of parks in any given districts, is discussed. The finding showed that there is no significant correlation between the amount of parks and the severity of floods. However, the amount of parks and severity of floods have a negative correlation, which means that having more parks indeed would be helpful to slightly reduce the severity of floods in any given district.